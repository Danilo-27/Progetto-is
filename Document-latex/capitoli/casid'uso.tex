\chapter{Modellazione dei Casi d'Uso}
\section{Attori e Casi d'Uso}
\begin{table}[!hbp]
	\centering
	\begin{tblr}{colspec=XX}
		\begin{minipage}[t]{\linewidth}
			\paragraph{Attori primari}
			\begin{itemize}
				\item UtenteNonRegistrato
				\item UtenteRegistrato
				\item Utente
				\item Amministratore
			\end{itemize}
		\end{minipage} &
		\begin{minipage}[t]{\linewidth}
			\paragraph{Attori secondari}
			\begin{itemize}
				\item SistemaGestioneAcquisti	
			\end{itemize}
		\end{minipage} \\
	\end{tblr}
\end{table}
\begin{table}[!hbp]
	\centering
	\begin{tblr}{colspec=XX}
		\begin{minipage}[t]{\linewidth}
			\paragraph{Casi d'uso}
			\begin{enumerate}
				\item Registrazione 
				\item Autenticazione
				\item RicercaEvento
				\item PubblicaEvento
				\item ConsultaInfromazioniEvento
				\item ConsultaCatalogoEventi
				\item PartecipazioneEvento
				\item AcquistoBiglietto
				\item ModificaDatiPersonali
				\item ConsultaStoricoBiglietti
				\item VisualizzaBiglietto
				\item ScaricaBiglietto
			\end{enumerate}
		\end{minipage} &
		\begin{minipage}[t]{\linewidth}
			\paragraph{Casi d'uso di inclusione}
			\begin{enumerate}
                \item ConsultaCatalogoEventi
			\end{enumerate}

		\end{minipage}
	\end{tblr}
\end{table}
\begin{table}[!ht]
\centering
\small
\begin{tblr}{
  colspec = {X[2,l] X[1.2,l] X[1.7,l] X[1.6,l] X[1.5,l]},
  hlines,
  row{1} = {font=\bfseries}
}
Caso d'uso & Attori Primari & Attori Secondari & Incl. / Ext. & Requisiti corrispondenti \\
Registrazione & UtenteNonRegistrato & -- & -- & \Req{rf}{01} \\
Aunteticazione & UtenteRegistrato & -- & -- & \Req{rf}{02} \\
RicercaEvento & UtenteRegistrato & -- & -- &\Req{rf}{08} \\
PubblicaEvento & Amministratore & -- & -- & \Req{rf}{06} \\
ConsultaInformazioniEvento & Amministratore & -- & Include: Consulta Catalogo Eventi & \Req{rf}{16}, \Req{rf}{17} \\
ConsultaCatalogoEventi & UtenteRegistrato & --  & -- & \Req{rf}{07} \\
PartecipazioneEvento& Utente & -- & -- & \Req{rf}{14} \\
AcquistaBiglietto & Utente & SistemaGestioneAcquisti & Include: ConsultaCatalogoEventi & \Req{rf}{09} \\
ModificaDatiPersonali & Utente & -- & -- & \Req{rf}{5} \\
ConsultaStoricoBiglietti & Utente & -- & --  & \Req{rf}{04} \\
VisualizzaBiglietto & Utente & -- & -- & \Req{rf}{11} \\
ScaricaBiglietto & Utente & -- & -- & \Req{rf}{12} \\
\end{tblr}
\end{table}

\section{Diagramma dei Casi d'Uso}
\begin{table}[!ht]
\centering
	\includegraphics[width=\linewidth]{assets/casid'uso/usd.png}
\end{table}	
\pagebreak
\section{Scenari}
\IncludeTable{capitoli/scenari/Registrazione}
\IncludeTable{capitoli/scenari/Autenticazione}
\IncludeTable{capitoli/scenari/ModificaDatiPersonali}
\IncludeTable{capitoli/scenari/RichiediCatalogoEventi}
\IncludeTable{capitoli/scenari/RicercaEvento}
\IncludeTable{capitoli/scenari/VisualizzaBiglietto}
\IncludeTable{capitoli/scenari/ConsultaStoricoBiglietti}
\IncludeTable{capitoli/scenari/ScaricaBiglietto}
\IncludeTable{capitoli/scenari/AcquistaBiglietto}
\IncludeTable{capitoli/scenari/PubblicaEvento}
\IncludeTable{capitoli/scenari/PartecipazioneEvento}
\IncludeTable{capitoli/scenari/ConsultaInformazioniEvento}
\newpage
\section{Diagramma delle Classi}
Di seguito riportiamo il diagramma delle classi di analisi.
\begin{figure}[!ht]
	\centering
	\includegraphics[width=0.8\linewidth]{assets/casid'uso/DiagramaDelleClassi.png}
	\caption{Diagramma delle classi di analisi}
\end{figure}


\begin{table}[h]
	\centering
	\begin{tblr}{
	  colspec = {X[0.6,c] X[0.6,c]},
	  width = 1\linewidth, 
	  hlines, vlines,
	  row{1} = {bg=gray!30, font=\bfseries}
	}
	RESPONSABILITÀ & CLASSE \\
	Registrazione & SistemaGestioneEventi \\
	Autenticazone & SistemagestioneEventi \\
	ModificaDati & Cliente \\
	VisualizzaStorico & Cliente \\
	CreazioneEvento & Amministratore \\
	ScaricaBiglietto & Biglietto \\
	VisualizzaBiglietto & Biglietto \\
	ValidaBiglietto & Evento \\
	AcquistoBiglietto & Evento \\
	PartecipazioneEvento & Evento \\
	ConsultaInformazioni & Evento \\
	AggiungiEvento & Catalogo \\
	filtra & catalogo \\


	\end{tblr}
\end{table}

\newpage

\section{Diagrammi di sequenza}

\subsection{Registrazione}

\begin{figure}[!ht]
	\centering
	La creazione del suddetto sequence diagram, sviluppato a partire dalla descrizione dello scenario del caso d’uso Accesso, ha fatto sorgere la necessità di definire un metodo, specifico per la classe SistemaGestioneEventi, verificaCredenziali(nomeUtente, password), privato, per consentire all’autonoleggio di verificare che le credenziali inserite dall’utente siano valide.
	\includegraphics[width=0.8\linewidth]{assets/casid'uso/Registrazione.png}
\end{figure}


%autenticazione
%registrazione
%ConsultaCatalogoEventi
%AcquistoBiglietto
%PartecipazioneEvento
%RicercaEvento
%ConsultaInformazioniEvento
%pubblicaevento