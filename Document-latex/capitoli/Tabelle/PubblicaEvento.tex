\section{PubblicaEvento}
\subsubsection*{Category Partition Testing}
\begin{table}[H]
	\centering
	\footnotesize
	\renewcommand{\arraystretch}{1.3}
	\begin{tabular}{|p{3cm}|p{3cm}|p{4cm}|p{5cm}|}
		\hline
		\textbf{Titolo} & \textbf{Descrizione} & \textbf{Data} & \textbf{Orario} & \textbf{Luogo} & \textbf{NumMassimoPartecipanti} & \textbf{Catalogo} \\
		\hline
		Stringa di lunghezza $\leq$ 50 \newline
		Stringa di lunghezza $>$ 50 \texttt{[ERROR]} \newline
		Stringa di lunghezza $<$ 0 \texttt{[ERROR]} \newline
		Stringa già memorizzata \texttt{[ERROR]} &

		Stringa di lunghezza $\leq$ 150 \newline %E' il caso di inserire come tipo un clob? già qua?
		Stringa di lunghezza $>$ 150 \texttt{[ERROR]} \newline
		Stringa di lunghezza $<$ 0 \texttt{[ERROR]} \newline

		Data con formato valido (gg-mm-aaaa) \newline
		Data con formato non valido \texttt{[ERROR]} \newline


		Orario con formato valido (oo-mm) \newline
		Orario con formato non valido \texttt{[ERROR]} \newline

		Stringa non contenente caratteri speciali \newline
		Stringa non contenente numeri \newline
		Stringa contenente caratteri speciali \texttt{[ERROR]} \newline
		Stringa contenente numeri \texttt{[ERROR]} \newline

		Intero di valore massimo 100 \newline
		Intero di valore $\gt$ 100 \texttt{[ERROR]} \newline

		Valido riferimento ad oggetto Catalogo \newline %non sono sicuro di questo
		Riferimento non valido ad oggetto Catalogo \texttt{[ERRORE]} \newline

		\hline
	\end{tabular}
	\caption{Category Partition Testing - PubblicaEvento}
\end{table}
\noindent Il numero di test da effettuarsi senza particolari vincoli è: $4 \cdot 5 = 20$.
\noindent Con i vincoli [ERROR], invece, il numero di test da eseguire per testare singolarmente i vincoli è 11 (3 per Email, 3 per Password).
\noindent Il numero di test risultante è 13: $(1 \cdot 1 \cdot 1 \cdot 2) + 6 = 8$.


\subsubsection*{Test Suite}

\begin{table}[H]
	\centering
	\footnotesize
	\begin{testsuite}{colspec = lXXXlXX}
	{Test \\ Case \\ ID} & Descrizione & Classi di Equivalenza Coperte & Pre-condizioni & Input & {Output \\ Attesi} & {Post-condizioni \\ Attese} \\
	1 & Tutti gli input validi &
	2 & Nome $>$ 45 caratteri & Nome $>$ 45 caratteri \texttt{[ERROR]}, Prezzo, Scorte e Prescrizione (sia \texttt{True} che \texttt{False}) validi & -- & {Nome: \dots \\ Prezzo: 16.10 \euro \\ Scorte: 23 \\ Prescrizione: \texttt{boolean}} &Nome troppo lungo & -- \\
	3 & Nome non specificato & Nome non specificato \texttt{[ERROR]}, Prezzo, Scorte e Prescrizione (sia \texttt{True} che \texttt{False}) validi & -- & {Nome: \\ Prezzo: 9.99 \euro \\ Scorte: 10 \\ Prescrizione: \texttt{boolean}} & Inserire un nome & -- \\
	4 & Prezzo $\leq 0$ (\euro) & Prezzo $\leq 0$ (\euro) \texttt{[ERROR]}, Nome, Scorte e Prescrizione (sia \texttt{True} che \texttt{False}) validi & -- & {Nome: Brufen \\ Prezzo: -8.00 \euro \\ Scorte: 200 \\ Prescrizione: \texttt{boolean}} & Inserire un prezzo $> 0$ \euro & -- \\
	5 & Scorte $ < 0$ & Scorte $<0$, Nome, Prezzo e Prescrizione (sia \texttt{True} che \texttt{False}) validi & - & {Nome: Macladin \\ Prezzo: 11.50 \euro \\ Scorte: -36 \\ Prescrizione: \texttt{boolean}} & Inserire scorte $ \geq 0 $ & -- \\
	6 & Nome già memorizzato & Nome già memorizzato \texttt{[ERROR]}, Prezzo, Scorte e Prescrizione (sia \texttt{True} che \texttt{False}) validi & Esiste già un farmaco con il nome inserito & {Nome: Cistalgan \\ Prezzo: 19.90 \euro \\ Scorte: 50 \\ Prescrizione : \texttt{boolean}} & Farmaco già esistente & -- \\
	\end{testsuite}
\end{table}