\section{PubblicaEvento}
\subsubsection*{Category Partition Testing}
\begin{table}[H]
	\centering
	\footnotesize
	\renewcommand{\arraystretch}{1.3}
	\begin{tabular}{|p{2.5cm}|p{2.5cm}|p{2.5cm}|p{2.5cm}|p{2.5cm}|p{2.5cm}|}
		\hline
		\textbf{Titolo} & \textbf{Descrizione} & \textbf{Data} & \textbf{Orario} & \textbf{Luogo} & \textbf{NumMassimo Partecipanti} \\
		\hline
		Stringa di lunghezza $\leq$ 50 \newline
		Stringa di lunghezza $>$ 50 \texttt{[ERROR]} \newline
		Stringa di lunghezza $<$ 0 \texttt{[ERROR]} \newline
		Stringa già memorizzata \texttt{[ERROR]} &

		Stringa di lunghezza $\leq$ 150 \newline
		Stringa di lunghezza $>$ 150 \texttt{[ERROR]} \newline
		Stringa di lunghezza $<$ 0 \texttt{[ERROR]} &

		Data con formato valido (gg-mm-aaaa) \newline
		Data con formato non valido \texttt{[ERROR]} &

		Orario con formato valido (oo-mm) \newline
		Orario con formato non valido \texttt{[ERROR]} &

		Stringa non contenente caratteri speciali \newline
		Stringa contenente caratteri speciali \texttt{[ERROR]} \newline
		Stringa contenente numeri \texttt{[ERROR]} &

		Intero di valore massimo 100 \newline
		Intero di valore $>$ 100 \texttt{[ERROR]} \\
		\hline
	\end{tabular}
	\caption{Category Partition Testing - PubblicaEvento}
\end{table}

\noindent Il numero di test da effettuarsi senza particolari vincoli è: $4 \cdot 3 \cdot 2 \cdot 2 \cdot 3 \cdot 2 = 288$.

\noindent Con i vincoli [ERROR], invece, il numero di test da eseguire per testare singolarmente i vincoli è 11 (3 per Titolo, 2 per Descrizione, 1 per Data, 1 per Orario, 2 per Luogo, 1 per NumMassimoPartecipanti).

\noindent Il numero di test risultante è 12: $(1 \cdot 1 \cdot 1 \cdot 1 \cdot 1 \cdot 1) + 11 = 12$.

\subsubsection*{Test Suite}

\begin{table}[H]
	\centering
	\tiny
	\renewcommand{\arraystretch}{1.4}
	\begin{tabular}{|c|p{2.5cm}|p{2.8cm}|p{1.5cm}|p{3.2cm}|p{2cm}|p{1.8cm}|}
		\hline
		\textbf{Test Case ID} & \textbf{Descrizione} & \textbf{Classi di Equivalenza Coperte} & \textbf{Pre-condizioni} & \textbf{Input} & \textbf{Output Attesi} & \textbf{Post-condizioni Attese} \\
		\hline
		1 & Tutti gli input validi & Tutti i parametri validi & Utente autenticato &
		Titolo: Concerto Rock, Descrizione: Evento musicale serale, Data: 15-06-2025, Orario: 20-30, Luogo: Teatro Comunale, NumMax: 50 &
		Evento pubblicato con successo & Evento salvato nel DB \\
		\hline
		2 & Titolo $>$ 50 caratteri & Titolo $>$ 50 caratteri \texttt{[ERROR]} & Utente autenticato &
		Titolo molto lungo (>50 char), altri parametri validi &
		Titolo troppo lungo & -- \\
		\hline
		3 & Titolo vuoto & Titolo con 0 caratteri \texttt{[ERROR]} & Utente autenticato &
		Titolo: "", altri parametri validi &
		Inserire un titolo & -- \\
		\hline
		4 & Titolo già esistente & Titolo già memorizzato \texttt{[ERROR]} & Utente autenticato &
		Titolo: Concerto Rock (esistente), altri parametri validi &
		Titolo già utilizzato & -- \\
		\hline
		5 & Descrizione $>$ 150 caratteri & Descrizione $>$ 150 caratteri \texttt{[ERROR]} & Utente autenticato &
		Descrizione molto lunga (>150 char), altri parametri validi &
		Descrizione troppo lunga & -- \\
		\hline
		6 & Descrizione vuota & Descrizione con 0 caratteri \texttt{[ERROR]} & Utente autenticato &
		Descrizione: "", altri parametri validi &
		Inserire una descrizione & -- \\
		\hline
		7 & Data formato non valido & Data con formato non valido \texttt{[ERROR]} & Utente autenticato &
		Data: 2025/06/15, altri parametri validi &
		Formato data non valido & -- \\
		\hline
		8 & Orario formato non valido & Orario con formato non valido \texttt{[ERROR]} & Utente autenticato &
		Orario: 20:30, altri parametri validi &
		Formato orario non valido & -- \\
		\hline
		9 & Luogo con caratteri speciali & Luogo con caratteri speciali \texttt{[ERROR]} & Utente autenticato &
		Luogo: Teatro\#Comunale, altri parametri validi &
		Caratteri non consentiti nel luogo & -- \\
		\hline
		10 & Luogo con numeri & Luogo con numeri \texttt{[ERROR]} & Utente autenticato &
		Luogo: Teatro123, altri parametri validi &
		Numeri non consentiti nel luogo & -- \\
		\hline
		11 & NumMassimo $>$ 100 & NumMassimo $>$ 100 \texttt{[ERROR]} & Utente autenticato &
		NumMax: 150, altri parametri validi &
		Numero massimo troppo alto & -- \\
		\hline
		12 & Utente non autenticato & Utente non autenticato & Utente non loggato &
		Tutti i parametri validi &
		Accesso negato & -- \\
		\hline
	\end{tabular}
	\caption{Test Suite - PubblicaEvento}
\end{table}