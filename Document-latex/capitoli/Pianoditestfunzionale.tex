\chapter{Piano di test funzionale}
Si intende progettare i casi di test funzionale con la tecnica del Category Partition Testing. 

\section{Registrazione}

\begin{table}[H]
\begin{tabularx}{\textwidth}{|X|X|X|X|}
\hline
\textbf{Nome} & \textbf{Cognome} & \textbf{Email} & \textbf{Password} \\
\hline
\begin{itemize}[leftmargin=*]
\item Stringa di lunghezza $\leq$ 20
\item Stringa di lunghezza $>$ 20 \texttt{[ERROR]}
\item Stringa vuota \texttt{[ERROR]}
\end{itemize} &
\begin{itemize}[leftmargin=*]
\item Stringa di lunghezza $\leq$ 30
\item Stringa di lunghezza $>$ 30 \texttt{[ERROR]}
\item Stringa vuota \texttt{[ERROR]}
\end{itemize} &
\begin{itemize}[leftmargin=*]
\item Stringa di lunghezza $\leq$ 50
\item Stringa con formato diverso da [esempio@dominio.estensione] \texttt{[ERROR]}
\item Stringa di lunghezza $>$ 50 \texttt{[ERROR]}
\item Stringa vuota \texttt{[ERROR]}
\item Stringa già memorizzata \texttt{[ERROR]}
\end{itemize} &
\begin{itemize}[leftmargin=*]
\item Stringa di lunghezza $\leq$ 40
\item Stringa di lunghezza $>$ 40 \texttt{[ERROR]}
\item Stringa vuota \texttt{[ERROR]}
\item Stringa senza caratteri speciali \texttt{[ERROR]}
\end{itemize} \\
\hline
\end{tabularx}
\end{table}

Il numero di test da effettuarsi senza particolari vincoli è: $3 \cdot 3 \cdot 5 \cdot 4 = 180$.

Introduciamo i vincoli \texttt{[ERROR]}. Il numero di test da eseguire per testare singolarmente i vincoli è 11 (2 per Nome, 2 per Cognome, 4 per Email, 3 per Password).

Il numero di test risultante è 11: $(1 \cdot 1 \cdot 1 \cdot 1) + 11 = 12$.

\IncludeTable{capitoli/Tabelle/Registrazione}
\section{Autenticazione}
\begin{table}[H]
    \centering
    \footnotesize
    \renewcommand{\arraystretch}{1.5}
    \begin{tabular}{|p{6cm}|p{6cm}|}
       \hline
       \textbf{Email} & \textbf{Password} \\
       \hline
       Stringa di lunghezza $\leq$ 50 \newline
	   Stringa con formato diverso da [esempio@dominio.estensione] \texttt{[ERROR]} \newline
	   Stringa di lunghezza $>$ 50 \texttt{[ERROR]} \newline
       Stringa di lunghezza $<$ 0 \texttt{[ERROR]} \newline
       Stringa non memorizzata \texttt{[ERROR]} &

       Stringa di lunghezza $\leq$ 40 \newline
       Stringa di lunghezza $>$ 40 \texttt{[ERROR]} \newline
       Stringa di lunghezza $=$ 0 \texttt{[ERROR]} \newline
       Stringa senza caratteri speciali \texttt{[ERROR]} \\
       \hline
    \end{tabular}
    \caption{Category Partition Testing - Autenticazione}
\end{table}

\noindent Il numero di test da effettuarsi senza particolari vincoli è: $5 \cdot 4 = 20$.

\noindent Con i vincoli [ERROR], invece, il numero di test da eseguire per testare singolarmente i vincoli è 7 (4 per Email, 3 per Password).

\noindent Il numero di test risultante è 7: $(1 \cdot 1 \cdot 1 \cdot 1) + 7 = 8$.

\IncludeTable{capitoli/Tabelle/Autenticazione}
\section{PubblicaEvento}
\begin{table}[H]
	\centering
	\footnotesize
	\renewcommand{\arraystretch}{1.3}
	\begin{tabular}{|p{2.5cm}|p{2.5cm}|p{2.5cm}|p{2.5cm}|p{2.5cm}|p{2.5cm}|}
		\hline
		\textbf{Titolo} & \textbf{Descrizione} & \textbf{Data} & \textbf{Orario} & \textbf{Luogo} & \textbf{Capienza} \\
		\hline
		Stringa di lunghezza $\leq$ 50 \newline
		Stringa di lunghezza $>$ 50 \texttt{[ERROR]} \newline
		Stringa di lunghezza $<$ 0 \texttt{[ERROR]} \newline
		Stringa già memorizzata \texttt{[ERROR]} &

		Stringa di lunghezza $\leq$ 150 \newline
		Stringa di lunghezza $>$ 150 \texttt{[ERROR]} \newline
		Stringa di lunghezza $<$ 0 \texttt{[ERROR]} &

		Data con formato valido (gg-mm-aaaa) \newline
		Data con formato non valido \texttt{[ERROR]} &

		Orario con formato valido (oo-mm) \newline
		Orario con formato non valido \texttt{[ERROR]} &

		Stringa non contenente caratteri speciali \newline
		Stringa contenente caratteri speciali \texttt{[ERROR]} \newline
		Stringa contenente numeri \texttt{[ERROR]} &

		Intero di valore massimo 100 \newline
		Intero di valore $>$ 100 \texttt{[ERROR]} \\
		\hline
	\end{tabular}
	\caption{Category Partition Testing - PubblicaEvento}
\end{table}

\noindent Il numero di test da effettuarsi senza particolari vincoli è: $4 \cdot 3 \cdot 2 \cdot 2 \cdot 3 \cdot 2 = 288$.

\noindent Con i vincoli [ERROR], invece, il numero di test da eseguire per testare singolarmente i vincoli è 11 (3 per Titolo, 2 per Descrizione, 1 per Data, 1 per Orario, 2 per Luogo, 1 per NumMassimoPartecipanti).

\noindent Il numero di test risultante è 12: $(1 \cdot 1 \cdot 1 \cdot 1 \cdot 1 \cdot 1) + 11 = 12$.

\IncludeTable{capitoli/Tabelle/PubblicaEvento}
\section{RicercaEvento}
\begin{table}[H]
	\centering
	\footnotesize
	\renewcommand{\arraystretch}{1.3}
	\begin{tabular}{|p{2.5cm}|p{2.5cm}|p{2.5cm}|}
		\hline
		\textbf{Titolo} & \textbf{Data} & \textbf{Luogo} \\
		\hline
		Stringa di lunghezza $\leq$ 50 \newline
		Stringa di lunghezza $>$ 50 \texttt{[ERROR]} \newline
		Stringa di lunghezza $<$ 0 \texttt{[ERROR]} \newline
		Stringa già memorizzata \texttt{[ERROR]} &
		
		Data con formato valido (gg-mm-aaaa) \newline
		Data con formato non valido \texttt{[ERROR]} &
		
		Stringa non contenente caratteri speciali \newline
		Stringa contenente caratteri speciali \texttt{[ERROR]} \newline
		Stringa contenente numeri \texttt{[ERROR]} \\
		\hline
	\end{tabular}
\end{table}

\noindent Il numero di test da effettuarsi senza particolari vincoli è:
$4 \cdot 2 \cdot 3 = 24$.

\noindent Con i vincoli \texttt{[ERROR]}, invece, il numero di test da eseguire per testare singolarmente i vincoli è 6 (3 per Titolo, 1 per Data, 2 per Luogo).

\noindent Il numero di test risultante è 7: $(1 \cdot 1 \cdot 1) + 6 = 7$.

\IncludeTable{capitoli/Tabelle/RicercaEvento}
\section{Acquista Biglietto}
\begin{table}[H]
	\centering
	\footnotesize
	\renewcommand{\arraystretch}{1.3}
	\begin{tabular}{|p{4cm}|p{4cm}|}
		\hline
		\textbf{Posti Disponibili} & \textbf{Dati Pagamento} \\
		\hline
		Posti Disponibili \newline
		Posti Esauriti \texttt{[ERROR]} &
		
		Dati completi e validi \newline
		Dati errati (es. carta scaduta) \texttt{[ERROR]} \newline
		Errore conferma dal sistema gestione acquisti \texttt{[ERROR]} \\
		\hline
	\end{tabular}
\end{table}
\noindent Il numero di test da effettuarsi senza particolari vincoli è:
$2 \cdot 3 \cdot 2 = 12$.

\noindent Con i vincoli \texttt{[ERROR]}, invece, il numero di test da eseguire per testare singolarmente i vincoli è 4 (1 per Posti disponibili, 2 per Dati pagamento, 1 per Sistema gestione acquisti).

\noindent Il numero di test risultante è 5: $(1 \cdot 1 \cdot 1) + 4 = 5$.
\IncludeTable{capitoli/Tabelle/AcquistoBiglietto}
\section{PartecipaEvento}
\begin{table}[H]
	\centering
	\footnotesize
	\renewcommand{\arraystretch}{1.3}
	\begin{tabular}{|p{4cm}|p{4cm}|}
		\hline
		\textbf{Codice Biglietto} & \textbf{Stato Biglietto} \\
		\hline
		Codice biglietto $=$ xxxx-123-ABC \newline
		Codice biglietto $\neq$ xxxx-123-ABC \texttt{[ERRORE]} &

		Stato biglietto $=$ [Non consumato] \newline
		Stato biglietto $=$ [Consumato] \texttt{[ERROR]} \\
		\hline
	\end{tabular}
\end{table}
\noindent Il numero di test da effettuarsi senza particolari vincoli è:
$2 \cdot 2 = 4$.

\noindent Con i vincoli \texttt{[ERROR]}, invece, il numero di test da eseguire per testare singolarmente i vincoli è 2 (1 per Codice Biglietto, 1 per Stato biglietto).

\noindent Il numero di test risultante è 5: $(1 \cdot 1 \cdot 1) + 2 = 3$.
\IncludeTable{capitoli/Tabelle/PartecipazioneEvento}