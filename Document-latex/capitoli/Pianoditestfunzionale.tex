\chapter{Piano di test funzionale}
Si intende progettare i casi di test funzionale con la tecnica del Category Partition Testing. 

\section{PartecipaEvento}
\begin{table}[H]
	\centering
	\footnotesize
	\renewcommand{\arraystretch}{1.3}
	\begin{tabular}{|p{4cm}|p{4cm}|}
		\hline
		\textbf{Codice Biglietto} & \textbf{Stato Biglietto} \\
		\hline
		Codice biglietto $=$ xxxx-123-ABC \newline
		Codice biglietto $\neq$ xxxx-123-ABC \texttt{[ERRORE]} &

		Stato biglietto $=$ [Non consumato] \newline
		Stato biglietto $=$ [Consumato] \texttt{[ERROR]} \\
		\hline
	\end{tabular}
\end{table}
\noindent Il numero di test da effettuarsi senza particolari vincoli è:
$2 \cdot 2 = 4$.

\noindent Con i vincoli \texttt{[ERROR]}, invece, il numero di test da eseguire per testare singolarmente i vincoli è 2 (1 per Codice Biglietto, 1 per Stato biglietto).

\noindent Il numero di test risultante è 5: $(1 \cdot 1 \cdot 1) + 2 = 3$.
\IncludeTable{capitoli/Tabelle/PartecipazioneEvento}