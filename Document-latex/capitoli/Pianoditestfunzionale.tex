\chapter{Piano di test funzionale}
Si intende progettare i casi di test funzionale con la tecnica del Category Partition Testing. 

\section{Registrazione}

\begin{table}[H]
\centering
\footnotesize
\renewcommand{\arraystretch}{1.3}
\begin{tabularx}{\textwidth}{|X|X|X|X|}
\hline
\textbf{Nome} & \textbf{Cognome} & \textbf{Email} & \textbf{Password} \\
\hline
\parbox[t]{\linewidth}{\begin{itemize}[leftmargin=*]
\item Stringa di lunghezza $\leq$ 20 \checkmark
\item Stringa di lunghezza $>$ 20 \texttt{[ERROR]}
\item Stringa vuota \texttt{[ERROR]}
\end{itemize}} &
\parbox[t]{\linewidth}{\begin{itemize}[leftmargin=*]
\item Stringa di lunghezza $\leq$ 30 \checkmark
\item Stringa di lunghezza $>$ 30 \texttt{[ERROR]}
\item Stringa vuota \texttt{[ERROR]}
\end{itemize}} &
\parbox[t]{\linewidth}{\begin{itemize}[leftmargin=*]
\item Stringa di lunghezza $\leq$ 50 \checkmark
\item Stringa con formato diverso da [esempio@dominio.estensione] \texttt{[ERROR]}
\item Stringa di lunghezza $>$ 50 \texttt{[ERROR]}
\item Stringa vuota \texttt{[ERROR]}
\item Stringa già memorizzata \texttt{[ERROR]}
\end{itemize}} &
\parbox[t]{\linewidth}{\begin{itemize}[leftmargin=*]
\item Stringa di lunghezza $\leq$ 40 \checkmark
\item Stringa di lunghezza $>$ 40 \texttt{[ERROR]}
\item Stringa vuota \texttt{[ERROR]}
\item Stringa senza caratteri speciali \texttt{[ERROR]}
\end{itemize}} \\
\hline
\end{tabularx}
\caption{Category Partition Testing - Registrazione}
\end{table}

Il numero di test da effettuarsi senza particolari vincoli è: $3 \cdot 3 \cdot 5 \cdot 4 = 180$.

Introduciamo i vincoli \texttt{[ERROR]}. Il numero di test da eseguire per testare singolarmente i vincoli è 11 (2 per Nome, 2 per Cognome, 4 per Email, 3 per Password).

Il numero di test risultante è 11: $(1 \cdot 1 \cdot 1 \cdot 1) + 11 = 12$.

\IncludeTable{capitoli/Tabelle/Registrazione}

\section{Autenticazione}
\begin{table}[H]
    \centering
    \footnotesize
    \renewcommand{\arraystretch}{1.3}
    \begin{tabularx}{\textwidth}{|X|X|}
       \hline
       \textbf{Email} & \textbf{Password} \\
       \hline
       \parbox[t]{\linewidth}{\begin{itemize}[leftmargin=*]
           \item Stringa di lunghezza $\leq$ 50 \checkmark
           \item Stringa con formato diverso da [esempio@dominio.estensione] \texttt{[ERROR]}
           \item Stringa di lunghezza $>$ 50 \texttt{[ERROR]}
           \item Stringa di lunghezza $<$ 0 \texttt{[ERROR]}
           \item Stringa non memorizzata \texttt{[ERROR]}
       \end{itemize}} &

       \parbox[t]{\linewidth}{\begin{itemize}[leftmargin=*]
           \item Stringa di lunghezza $\leq$ 40 \checkmark
           \item Stringa di lunghezza $>$ 40 \texttt{[ERROR]}
           \item Stringa di lunghezza $=$ 0 \texttt{[ERROR]}
           \item Stringa senza caratteri speciali \texttt{[ERROR]}
       \end{itemize}} \\
       \hline
    \end{tabularx}
    \caption{Category Partition Testing - Autenticazione}
\end{table}

\noindent Il numero di test da effettuarsi senza particolari vincoli è: $5 \cdot 4 = 20$.

\noindent Con i vincoli [ERROR], invece, il numero di test da eseguire per testare singolarmente i vincoli è 7 (4 per Email, 3 per Password).

\noindent Il numero di test risultante è 7: $(1 \cdot 1 \cdot 1 \cdot 1) + 7 = 8$.

\IncludeTable{capitoli/Tabelle/Autenticazione}

\section{PubblicaEvento}
\begin{table}[H]
	\centering
	\footnotesize
	\renewcommand{\arraystretch}{1.3}
	\begin{tabularx}{\textwidth}{|X|X|X|X|X|X|X|}
		\hline
		\textbf{Titolo} & \textbf{Descrizione} & \textbf{Data} & \textbf{Orario} & \textbf{Luogo} & \textbf{Capienza} & \textbf{Costo} \\
		\hline

		\parbox[t]{\linewidth}{\begin{itemize}[leftmargin=*]
			\item Stringa di lunghezza $\leq$ 50 \checkmark
			\item Lunghezza $>$ 50 \texttt{[ERROR]}
			\item Lunghezza $\leq$ 0 \texttt{[ERROR]}
			\item Stringa già esistente nel sistema \texttt{[ERROR]}
		\end{itemize}} &

		\parbox[t]{\linewidth}{\begin{itemize}[leftmargin=*]
			\item Stringa di lunghezza $\leq$ 150 \checkmark
			\item Lunghezza $>$ 150 \texttt{[ERROR]}
			\item Lunghezza $\leq$ 0 \texttt{[ERROR]}
		\end{itemize}} &

		\parbox[t]{\linewidth}{\begin{itemize}[leftmargin=*]
			\item Formato valido e data futura (gg-mm-aaaa) \checkmark
			\item Formato valido e data odierna \checkmark
			\item Formato valido ma data non futura \texttt{[ERROR]}
			\item Formato non valido \texttt{[ERROR]}
		\end{itemize}} &

		\parbox[t]{\linewidth}{\begin{itemize}[leftmargin=*]
			\item Formato valido (oo-mm) \checkmark
			\item Formato non valido \texttt{[ERROR]}
		\end{itemize}} &

		\parbox[t]{\linewidth}{\begin{itemize}[leftmargin=*]
			\item Solo lettere e spazi \checkmark
			\item Contiene caratteri speciali \texttt{[ERROR]}
			\item Contiene numeri \texttt{[ERROR]}
		\end{itemize}} &

		\parbox[t]{\linewidth}{\begin{itemize}[leftmargin=*]
			\item Intero $\leq$ 500 \checkmark
			\item Intero $>$ 500 \texttt{[ERROR]}
			\item Intero $\leq$ 0 \texttt{[ERROR]}
			\item Formato non numerico \texttt{[ERROR]}
		\end{itemize}} &

		\parbox[t]{\linewidth}{\begin{itemize}[leftmargin=*]
			\item Numero $\geq$ 0 \checkmark
			\item Numero negativo \texttt{[ERROR]}
			\item Formato non numerico \texttt{[ERROR]}
		\end{itemize}} \\
		\hline
	\end{tabularx}
	\caption{Category Partition Testing - PubblicaEvento}
\end{table}

\noindent Il numero di test da effettuarsi senza particolari vincoli è: $4 \cdot 3 \cdot 3 \cdot 2 \cdot 3 \cdot 4 \cdot 3 = 2592$.

\noindent Con i vincoli [ERROR], invece, il numero di test da eseguire per testare singolarmente i vincoli è 16 (3 per Titolo, 2 per Descrizione, 2 per Data, 1 per Orario, 2 per Luogo, 3 per Capienza, 3 per Costo).

\noindent Il numero di test risultante è 16: $(1 \cdot 1 \cdot 2\cdot 1 \cdot 1 \cdot 1) + 16 = 18$.

\IncludeTable{capitoli/Tabelle/PubblicaEvento}


\section{RicercaEvento}
\begin{table}[H]
	\centering
	\footnotesize
	\renewcommand{\arraystretch}{1.3}
	\begin{tabularx}{\textwidth}{|X|X|X|}
		\hline
		\textbf{Titolo} & \textbf{Data} & \textbf{Luogo} \\
		\hline
		\parbox[t]{\linewidth}{\begin{itemize}[leftmargin=*]
			\item Stringa di lunghezza $\leq$ 50 \checkmark
			\item Lunghezza $>$ 50 \texttt{[ERROR]}
			\item Lunghezza $\leq$ 0 \texttt{[ERROR]}
		\end{itemize}} &

		\parbox[t]{\linewidth}{\begin{itemize}[leftmargin=*]
			\item Formato valido e data futura (gg-mm-aaaa) \checkmark
			\item Formato valido e data odierna \checkmark
			\item Formato valido ma data non futura \texttt{[ERROR]}
			\item Formato non valido \texttt{[ERROR]}
		\end{itemize}} &
		\parbox[t]{\linewidth}{\begin{itemize}[leftmargin=*]
			\item Contiene solo lettere e spazi \checkmark
			\item Contiene caratteri speciali \texttt{[ERROR]}
			\item Contiene numeri \texttt{[ERROR]}
		\end{itemize}} \\
		\hline
	\end{tabularx}
	\caption{Category Partition Testing - RicercaEvento}
\end{table}
\noindent Il numero di test da effettuarsi senza particolari vincoli è:
$3 \cdot 4 \cdot 3 = 36$.

\noindent Con i vincoli \texttt{[ERROR]}, invece, il numero di test da eseguire per testare singolarmente i vincoli è 6 (2 per Titolo, 2 per Data, 2 per Luogo).

\noindent Il numero di test risultante è 8: $(1 \cdot 2 \cdot 1) + 6 = 8$.
\IncludeTable{capitoli/Tabelle/RicercaEvento}

\section{Acquista Biglietto}
\begin{table}[H]
    \centering
    \footnotesize
    \renewcommand{\arraystretch}{1.3}
    \begin{tabularx}{\textwidth}{|X|X|X|X|X|}
        \hline
        \textbf{Evento} & \textbf{Dati Pagamento}  & \textbf{Saldo Disponibile} & \textbf{Stato Cliente} \\
        \hline

        \parbox[t]{\linewidth}{\begin{itemize}[leftmargin=*]
            \item Posti disponibili \checkmark
            \item Posti esauriti \texttt{[ERROR]}
        \end{itemize}} &

        \parbox[t]{\linewidth}{\begin{itemize}[leftmargin=*]
            \item Dati completi e validi \checkmark
            \item Numero carta non contiene solo numeri \texttt{[ERROR]}
            \item Data scadenza non valida \texttt{[ERROR]}
            \item Carta Scaduta \texttt{[ERROR]}
        \end{itemize}} &

        \parbox[t]{\linewidth}{\begin{itemize}[leftmargin=*]
            \item Saldo sufficiente \checkmark
            \item Saldo non sufficiente \texttt{[ERROR]}
        \end{itemize}} &

        \parbox[t]{\linewidth}{\begin{itemize}[leftmargin=*]
            \item Cliente non ha acquistato ancora per questo evento \checkmark
            \item Cliente ha già acquistato un biglietto per l’evento \texttt{[ERROR]}
        \end{itemize}} \\
        \hline
    \end{tabularx}
    \caption{Category Partition Testing - Acquisto Biglietto}
\end{table}
\noindent Il numero di test da effettuarsi senza particolari vincoli è:
$2 \cdot 4 \cdot 2 \cdot 2 = 32$.

\noindent Con i vincoli \texttt{[ERROR]}, invece, il numero di test da eseguire per testare singolarmente i vincoli è 6 (1 per Evento ,3 per Dati pagamento, 1 per Saldo Disponibile , 1 Stato Cliente ).

\noindent Il numero di test risultante è 5: $(1 \cdot 1 \cdot 1 \cdot 1) + 6= 7$.
\IncludeTable{capitoli/Tabelle/AcquistoBiglietto}

\section{PartecipaEvento}
\begin{table}[H]
    \centering
    \footnotesize
    \renewcommand{\arraystretch}{1.3}
    \begin{tabularx}{\textwidth}{|X|X|}
        \hline
        \textbf{Codice Biglietto} & \textbf{Stato Biglietto} \\
        \hline
        \parbox[t]{\linewidth}{\begin{itemize}[leftmargin=*]
            \item Codice biglietto corretto (es. xxxx-123) \checkmark
            \item Codice biglietto errato/non esistente \texttt{[ERROR]}
        \end{itemize}} &

        \parbox[t]{\linewidth}{\begin{itemize}[leftmargin=*]
            \item Stato = Non consumato \checkmark
            \item Stato = Consumato \texttt{[ERROR]} 
        \end{itemize}} \\
        \hline
    \end{tabularx}
    \caption{Category Partition Testing - PartecipaEvento}
\end{table}
\noindent Il numero di test da effettuarsi senza particolari vincoli è:
$2 \cdot 2 = 4$.

\noindent Con i vincoli \texttt{[ERROR]}, invece, il numero di test da eseguire per testare singolarmente i vincoli è 2 (1 per Codice Biglietto, 1 per Stato biglietto).

\noindent Il numero di test risultante è 5: $(1 \cdot 1 \cdot 1) + 2 = 3$.
\IncludeTable{capitoli/Tabelle/PartecipazioneEvento}