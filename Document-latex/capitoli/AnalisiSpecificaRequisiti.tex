
\chapter{Analisi e specifica dei requisiti}
\section{Analisi nomi-verbi}


\Func{Il sistema consente la registrazione} di \Actor{utenti}, che devono fornire \Attr{nome, cognome, indirizzo e-mail e password}. \Func{Ogni utente dispone di un }\Class{profilo personale}, accessibile tramite \Func{autenticazione,} dove \Func{pu\`{o} visualizzare e gestire le informazioni del proprio account, modificare i dati personali}, e consultare lo \Attr{storico dei biglietti} acquistati ed opzionalmente \Attr{la propria immagine del profilo}. Ogni profilo mostra opzionalmente anche il \Attr{numero totale di eventi a cui l’utente ha partecipato}.

\Actor{Gli amministratori} della piattaforma possono \Func{creare nuovi eventi}, specificando per ciascuno \Attr{titolo, descrizione, data, orario, luogo e numero massimo di partecipanti}. \Func{Gli eventi pubblicati sono consultabili dagli} \ClassActor{utenti registrati} \Func{tramite un catalogo eventi, filtrabile opzionalmente per data o localit\`{a}}.

Durante il processo di \Func{acquisto}, l’utente seleziona un evento e riceve un \Class{biglietto elettronico}, identificato da un \Attr{codice univoco}. Il biglietto contiene: \Attr{nome dell’evento, data, orario, nome del partecipante e codice identificativo}. \Func{I biglietti possono essere scaricati o visualizzati} direttamente dal profilo utente.

Nel giorno dell’evento, l’utente pu\`{o} accedere a una apposita interfaccia grafica pensata per il \Func{controllo degli accessi}. In questa interfaccia \Func{gli viene presentato l’elenco di tutti gli eventi previsti per la data odierna}. L’utente \Func{seleziona l’evento a cui intende partecipare} e inserisce il \Attr{codice del biglietto} precedentemente ricevuto. \Func{Il sistema}, a questo punto, \Func{effettua una serie di verifiche}: controlla che il codice del biglietto esista e sia effettivamente associato all’evento selezionato, che la data indicata sul biglietto coincida con quella odierna e che il biglietto non sia gi\`{a} stato utilizzato. Se tutte le condizioni risultano verificate, il sistema \Func{autorizza l’accesso} e marca il biglietto come “consumato”. In caso contrario, viene restituito un messaggio di errore esplicativo che impedisce l’ingresso.

\Func{Il sistema mantiene traccia in tempo reale delle persone presenti a ciascun evento}, aggiornando dinamicamente il numero di ingressi effettuati. Gli amministratori possono \Func{accedere a un pannello di gestione per ogni evento. Per gli eventi odierni, il sistema consente di visualizzare non solo il numero di utenti registrati, ma anche l’elenco aggiornato degli utenti effettivamente presenti in quel momento. Per gli eventi passati, invece, l’amministratore potr\`{a} accedere anche al numero totale di partecipanti che hanno avuto accesso}, senza possibilit\`{a} di consultarne i nomi.

\Func{L’applicazione deve essere accessibile via web da dispositivi desktop e mobili}, offrire un’interfaccia grafica chiara e intuitiva, e \Func{implementare meccanismi di sicurezza per la protezione dei dati personali, l’autenticazione degli utenti e l’integrit\`{a} dei biglietti elettronici}.

\bigskip
\vspace{1cm}
\noindent\textbf{\underline{LEGENDA}}\\[0.5em]
\begin{tabular}{ll}
\sethlcolor{ColorClass}\hl{\textbf{Classe}} \\
\sethlcolor{ColorAttr}\hl{\textbf{Attributo}} \\
\sethlcolor{ColorFunc}\hl{\textbf{Funzionalit\`{a}}}\\
\sethlcolor{ColorActor}\hl{\textbf{Attore}}\\
\sethlcolor{ColorClassActor}\hl{\textbf{Classe-Attore}}
\end{tabular}
/Users/danilocioffi/Documents/progettoIS/DiagrammaCasiD\'uso.png

\newpage
\section{Revisione dei Requisiti}

\begin{enumerate}[]
    \item Il sistema deve consentire ad un utente non registrato di registrarsi
    \item La registrazione consiste nell’inserire nome, cognome, indirizzo e-mail e password
    \item Il sistema deve offrire una funzionalità di autenticazione
    \item Il sistema deve gestire un profilo personale per ogni utente
    \item Il sistema deve tener traccia per ogni utente del numero totale di eventi a cui l’utente ha partecipato e un' immagine di profilo
    \item Il sistema consente di visualizzare lo storico dei biglietti acquisati dall'utente
    \item Il sistema offre una funzionalità di modifica dei dati personali (nome,cognome,indirizzo e-mail,password,ImmagineProfilo)
    \item Il sistema deve consentire agli amministratori la creazione di eventi
    \item Degli eventi si vuole memorizzare titolo, data, orario, luogo e numero massimo di partecipanti
    \item Il sistema deve offrire un catalogo eventi consultabile dagli utenti registrati
    \item Il sistema deve fornire una funzionalità di ricerca di un evento per nome, data o località.
    \item Il sistema deve offrire la funzionalità di acquisto dei biglietti
    \item Per l’acquisto dei biglietti, il sistema si interfaccia con un sistema di gestione degli acquisti esterno al sistema
    \item Ogni biglietto elettronico deve avere un codice identificativo univoco e contenere il nome dell’evento e il nome del partecipante
    \item il sistema deve offrirre la possibilità all'utente di visualizzare un biglietto acquistato
    \item Il sistema deve offrire la possibilità all'utente di scaricare un biglietto acquistato
    \item Il sistema deve consentire all’utente di confermare la partecipazione ad un evento tramite l’inserimento del codice di un biglietto acquistato
    \item Un biglietto marcato come consumato non può essere più essere riutilizzato
    \item Un utente registrato durante la fase di acquisto può comprare un solo biglietto per un dato evento
    \item Il sistema deve tener traccia dinamicamente del numero di partecipazioni per ogni evento
    \item Il sistema deve permettere agli amministratori di visualizzare il numero di partecipanti presenti per ogni evento in corso
    \item Il sistema deve fornire, per gli eventi passati, il numero totale di partecipanti che hanno avuto accesso
    \item Il sistema deve implementare meccanismi di sicurezza per la protezione dei dati personali e per l’autenticazione degli utenti
    \item Il sistema deve offrire un’interfaccia grafica chiara e intuitiva
    \item Il sistema deve garantire l’integrità dei biglietti elettronici
    \item Il sistema deve essere accessibile da dispositivi mobili e desktop
\end{enumerate}

\section{Glossario dei termini}



\begin{tblr}{
	colspec = lXl,
	hlines, vlines,
    row{1} = {bg=gray!30, font=\bfseries}
}
\hline
	Termine & Descrizione & Sinonimi \\
\hline    
Amministratore & Amministratore della piattaforma che si occupa della gestione degli eventi & \\
Biglietto elettronico & Biglietto acquistabile e utilizzare per partecipare all'evento a cui è associato & \\
Catalogo eventi & Catalogo che contiene tutti gli eventi pubblicati dagli amministratori & \\
Utente non registrato & Una persona che intende registrarsi presso il sistema & \\
Utente registrato & Un Utente che si è registrato presso il sistema & \\
Cliente & Utente registrato che acquista o partecipa a eventi. Nei diagrammi dei casi d’uso è rappresentato dall’attore "Utente"\\
\end{tblr}



\section{Classificazione dei Requisiti}

\subsection{Requisiti Funzionali}


\begin{tblr}{
	colspec = lXl,
	hlines, vlines,
	row{1} = {bg=gray!30, font=\bfseries}
}
\hline
ID & Requisito & Origine \\
\hline
\Req{rf}{01} & Il sistema offre la possibilità all’utente di registrarsi & 1 \\
\Req{rf}{02} & Il sistema deve offrire una funzionalità di autenticazione & 3 \\
\Req{rf}{03} & Il sistema deve gestire un profilo personale per ogni utente registrato & 4 \\
\Req{rf}{04} & Il sistema consente di visualizzare lo storico dei biglietti agli utenti autenticati & 6 \\
\Req{rf}{05} & Il sistema offre una funzionalità di modifica dei dati personali agli utenti autenticati & 7 \\
\Req{rf}{06} & Il sistema deve consentire agli amministratori la creazione di eventi & 8 \\
\Req{rf}{07} & Il sistema deve offrire un catalogo eventi consultabile dall'utente autenticato & 10 \\
\Req{rf}{08} & Il sistema deve fornire una funzionalità di ricerca di un evento per nome, data o località. & 11 \\
\Req{rf}{09} & Il sistema deve offrire la funzionalità di acquisto dei biglietti & 12 \\
\Req{rf}{10} & Il sistema offre la possibilità di visualizzare il biglietto acquistato dal profilo utente & 14 \\
\Req{rf}{11} & Il sistema offre la possibilità di scaricare il biglietto acquistato dal profilo utente & 15 \\
\Req{rf}{12} & Il sistema deve consentire all’utente di confermare la partecipazione ad un evento tramite l’inserimento del biglietto & 16\\
\Req{rf}{13} & Il sistema deve tener traccia dinamicamente del numero di partecipazioni per ogni evento & 19 \\
\Req{rf}{14} & Il sistema deve permettere agli amministratori di visualizzare il numero di partecipanti presenti per ogni evento in corso & 20 \\
\Req{rf}{15} & Il sistema deve fornire, per gli eventi passati, il numero totale di partecipanti che hanno avuto accesso & 21 \\
\end{tblr}

\subsection{Requisiti sui Dati}

\begin{tblr}{
	colspec = lXl,
	hlines, vlines,
	row{1} = {bg=gray!30, font=\bfseries}
	}
\hline
ID & Requisito & Origine \\
\hline
\Req{RD}{01} & La registrazione consiste nell’inserire nome, cognome, indirizzo e-mail e password & 2 \\
\Req{RD}{02} & Degli eventi si vuole memorizzare titolo, data, orario, luogo e numero massimo di partecipanti & 9 \\
\Req{RD}{03} & Ogni biglietto elettronico deve avere un codice identificativo univoco e contenere: nome dell’evento, data, orario, nome del partecipante e codice identificativo & 13 \\


\end{tblr}


\subsection{Vincoli/Altri Requisiti}

\begin{tblr}{
	colspec = lXl,
	hlines, vlines,
	row{1} = {bg=gray!30, font=\bfseries}
	}
    \hline
ID & Requisito & Origine \\
\hline
\Req{V}{01} & Per l’acquisto del biglietto, deve essere disponibile un sistema di gestione degli acquisti esterno al sistema & 12 \\
\Req{V}{02} & Un biglietto marcato come consumato non può essere più riutilizzato & 17 \\
\Req{V}{03} & Un utente registrato durante la fase di acquisto può comprare un solo biglietto & 18 \\
\Req{V}{04} & Il sistema deve offrire un’interfaccia grafica chiara e intuitiva & 23 \\
\Req{V}{05} & Per l’acquisto dei biglietti, il sistema si interfaccia con un sistema di gestione degli acquisti esterno al sistema & 13 \\
\Req{RNF}{01} & Il sistema deve implementare meccanismi di sicurezza per la protezione dei dati personali & 22 \\
\Req{RNF}{02} & Il sistema deve garantire l’integrità dei biglietti elettronici & 24 \\
\Req{RNF}{03} & Il sistema deve essere accessibile da dispositivi mobili e desktop & 25 \\
    
\end{tblr}
