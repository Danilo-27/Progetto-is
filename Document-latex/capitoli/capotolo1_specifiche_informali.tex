\chapter{Specifiche Informali}
Si intende sviluppare un sistema software per la gestione della vendita di biglietti per eventi, con funzionalità di controllo accessi in fase di partecipazione. Il sistema è destinato sia agli utenti finali (partecipanti) sia agli amministratori che organizzano e gestiscono gli eventi.\\

\noindent Il sistema consente la registrazione di utenti, che devono fornire nome, cognome, indirizzo e-mail e password. Ogni utente dispone di un profilo personale, accessibile tramite autenticazione, dove può visualizzare e gestire le informazioni del proprio account, modificare i dati personali, e consultare lo storico dei biglietti acquistati ed opzionalmente la propria immagine del profilo. Ogni profilo mostra opzionalmente anche il numero totale di eventi ha cui l’utente ha partecipato.\\

\noindent Gli amministratori della piattaforma possono creare nuovi eventi, specificando per ciascuno titolo, descrizione, data, orario, luogo e numero massimo di partecipanti. Gli eventi pubblicati sono consultabili dagli utenti registrati tramite un catalogo eventi, filtrabile opzionalmente per data o località.\\

\noindent Durante il processo di acquisto, l’utente seleziona un evento e riceve un biglietto elettronico, identificato da un codice univoco. Il biglietto contiene: nome dell’evento, data, orario, nome del partecipante e codice identificativo. I biglietti possono essere scaricati o visualizzati direttamente dal profilo utente.\\

\noindent Nel giorno dell’evento, l’utente può accedere a una apposita interfaccia grafica pensata per il controllo degli accessi. In questa interfaccia gli viene presentato l’elenco di tutti gli eventi previsti per la data odierna. L’utente seleziona l’evento a cui intende partecipare e inserisce il codice del biglietto precedentemente ricevuto. Il sistema, a questo punto, effettua una serie di verifiche: controlla che il codice del biglietto esista e sia effettivamente associato all’evento selezionato, che la data indicata sul biglietto coincida con quella odierna e che il biglietto non sia già stato utilizzato. Se tutte le condizioni risultano verificate, il sistema autorizza l’accesso e marca il biglietto come “consumato”. In caso contrario, viene restituito un messaggio di errore esplicativo che impedisce l’ingresso.\\

\noindent Il sistema mantiene traccia in tempo reale delle persone presenti a ciascun evento, aggiornando dinamicamente il numero di ingressi effettuati. Gli amministratori possono accedere a un pannello di gestione per ogni evento. Per gli eventi odierni, il sistema consente di visualizzare non solo il numero di utenti registrati, ma anche l’elenco aggiornato degli utenti effettivamente presenti in quel momento. Per gli eventi passati, invece, l’amministratore potrà accedere unicamente al numero totale di partecipanti che hanno avuto accesso, senza possibilità di consultare i nomi.\\

\noindent L’applicazione deve essere accessibile via web da dispositivi desktop e mobili, offrire un’interfaccia grafica chiara e intuitiva, e implementare meccanismi di sicurezza per la protezione dei dati personali, l’autenticazione degli utenti e l’integrità dei biglietti elettronici.


